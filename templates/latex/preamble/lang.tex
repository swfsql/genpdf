


\pdfminorversion=6
\pdfcompresslevel=9
\pdfobjcompresslevel=3

% ---------------------------------------------------------


\usepackage[utf8]{inputenc}

% TODO: first check if latin is to be used
\usepackage{libertine}
\usepackage{libertinust1math}

\usepackage[T1]{fontenc}




% useful commands:

% TODO: all languages that are to be used must be declared
% at the beggining
% and the last one is the main one

% \selectlanguage{czech}
% % changes the "complete setup" to the language "from now on"
% % for a kind of less-instrudive, you may do a:
% {\selectlanguage{czech} blablabla }\selectlanguage{english}

% \selectlanguage{english}
% % changes the "complete setup" to the language "from now on"

% % if more power is needed, such as changing left->right to right->left,
% % use something like
% \begin{otherlanguage}{hebrew}
%  blablabla
% \end{otherlanguage}
%
%

% % if only a small snippet is desired, without changing "too much", then
%
% \begin{otherlanguage*}{thai}
%  blablabla
% \end{otherlanguage*}
% % you may also enclose with braces, so it's more "local"
%
% % or else,
%
% \foreignlanguage{thai}{blablabla}
% % this does not change names nor dates
% % does not really require the language to be loaded
% 






% ---------------------------------------------------------






\usepackage[japanese]{babel}

\usepackage{polyglossia}%



\usepackage{fontspec}%
\usepackage[space,CJKchecksingle]{xeCJK} % guess the CJKnumber, etc have no effect
\usepackage[Latin, Thai]{ucharclasses}% TODO: add CJK

\usepackage{xltxtra}

% installed fonts: 
% fandol and wadalab (cjk-fonts)
% Linux Libertine OpenType
% IPAex font OTF http://ipafont.ipa.go.jp/
% tlwg OTF fonts https://www.hawaii.edu/thai/thaifonts/ 
%  ftp://linux.thai.net/pub/thailinux/software/fonts-tlwg/fonts/


\setCJKmainfont[Script=CJK]{MS Mincho} % for \rmfamily
\setCJKsansfont[Script=CJK]{MS Gothic} % for \sffamily
\setCJKmonofont[Script=CJK]{MS Gothic}
\XeTeXlinebreaklocale "ja"  %% Zeilenumbruch für japanische Texte
\XeTeXlinebreakskip=0em plus 0.1em minus 0.01em
\usepackage{setspace}

%\setCJKmainfont[
%  BoldFont=WenQuanYi Zen Hei,
%  ItalicFont=AR PL KaitiM GB]
%  {AR PL SungtiL GB}
%\setCJKsansfont{Noto Sans CJK SC}
%\setCJKmonofont{cwTeXFangSong}

\setmainlanguage[numerals=thai]{->{{def_lang.set_lang}}}% thai
\XeTeXlinebreaklocale "th_TH"
\XeTeXlinebreakskip = 0pt plus 1pt  

\setmainlanguage[changecounternumbering=true]{->{{def_lang.set_lang}}}% bengali

\setmainlanguage[script=fraktur]{->{{def_lang.set_lang}}}% german
 % burmese
\XeTeXlinebreaklocale "my"
\XeTeXlinebreakskip = 0pt plus 0.1pt  

\setmainlanguage{->{{def_lang.set_lang}}}%




\setotherlanguage{english}%





    
        

\setCJKmainfont[Script=CJK]{MS Mincho} % for \rmfamily
\setCJKsansfont[Script=CJK]{MS Gothic} % for \sffamily
\setCJKmonofont[Script=CJK]{MS Gothic}
\setotherlanguage{->{{lang.set_lang}}}%
        
        
        
% \setotherlanguage{->{{lang.set_lang}}}%
        
    


\defaultfontfeatures{Ligatures=TeX}%


\newfontfamily\defaultfont{Linux Libertine O}% Code2000
\newfontfamily\latinfont{Linux Libertine O}%
\newfontfamily\thaifont[Script=Thai]{Norasi}%
\newfontfamily{\thaifonttt}[Script=Thai]{TlwgMono}%
\newfontfamily\cyrillicfont[Script=Cyrillic]{Linux Libertine O}% not tested
\newfontfamily{\cyrillicfonttt}[Script=Cyrillic]{Linux Libertine O}% not tested
%\newfontfamily{\cjkfont}{HAN NOM A}
%\newfontfamily{\unifiedCJKfont}{SimSun-ExtB}
%\newCJKfontfamily\japanesefont{IPAex明朝}
%\setCJKfamilyfont{cjk-vert}[Script=CJK,RawFeature=vertical]{IPAex明朝}
%\newCJKfontfamily\koreafont{Baekmuk Batang}

\setDefaultTransitions{\defaultfont}{}

\setTransitionsForLatin{\latinfont}{}
%\setTransitionsForCJK{\cjkfont}{}
%\setTransitionsForJapanese{\japanesefont}{}
%\setTransitionTo{CJKUnifiedIdeographsExtensionB}{\unifiedCJKfont}
\setTransitionTo{Thai}{\thaifont}