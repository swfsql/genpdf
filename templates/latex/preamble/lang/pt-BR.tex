% allows utf8 reading from the source-code
\usepackage[utf8]{inputenc}
\usepackage[T1]{fontenc}

% font usage

	% print settings

    % linux libertine serif font
    \usepackage{libertine}
    \usepackage{libertinust1math}

	% digital settings

    % linux libertine sans-serif font (biolinum)
    \usepackage{libertine}
    \usepackage{libertinust1math}
    \renewcommand*\familydefault{\sfdefault}

    % % another option is carlito, but it's too dense
    % % carlito sans-serif font
    % % pretty similar to linux-libertine font
    % % and it's also easier to read than biolinum
    % \usepackage[sfdefault,lf]{carlito}
    % \renewcommand*\oldstylenums[1]{\carlitoOsF #1}


	\invalidReader,



    \TODOSecundaryForPortuguese

    \usepackage[brazil]{babel}
    \usepackage[babel=true,kerning=true]{microtype}

    % TODO: idk if this could be used in any language,
    % so its here..
    \usepackage{lettrine}

