
% usado para a capa...
\usepackage{tikz}
\usetikzlibrary{positioning}
\usepackage{varwidth}
\pgfdeclarelayer{bg}    % declare background layer
\pgfdeclarelayer{front} 
\pgfsetlayers{bg,main,front}  % set the order of the layers (main is the standard layer)



\usetikzlibrary{calc}
\usepackage{graphicx}
\usepackage{calc}
\usepackage{ifthen}


%\DeclareUrlCommand\Hurl{->%
%  \def\UrlLeft{\langle}%
%  \def\UrlRight{\rangle}%
%}
%\renewcommand*{\UrlFont}{\ttfamily\scriptsize\relax}



\newlength{\oH}
\newlength{\oW}
\newlength{\rH}
\newlength{\rW}
\newlength{\cH}
\newlength{\cW}
\newcommand\ClipImage[3]{->% width, height, image
\settototalheight{\oH}{\includegraphics{->#3}}%
\settowidth{\oW}{\includegraphics{->#3}}%
\setlength{\rH}{\oH * \ratio{->#1}{\oW}}%
\setlength{\rW}{\oW * \ratio{->#2}{\oH}}%
\ifthenelse{\lengthtest{\rH < #2}}{->%
    \setlength{\cW}{(\rW-#1)*\ratio{\oH}{->#2}}%
    \adjincludegraphics[height=#2,clip,trim=0 0 \cW{} 0]{->#3}%
}{->%
    \setlength{\cH}{(\rH-#2)*\ratio{\oW}{->#1}}%
    \adjincludegraphics[width=#1,clip,trim=0 \cH{} 0 0]{->#3}%
}%
}




% para listar o link da discussão e a data:
\makeatletter
\def\blfootnote{\xdef\@thefnmark{->$ \sim $}\@footnotetext}
\makeatother
% https://tex.stackexchange.com/questions/250221/supressing-the-footnote-number




\usepackage{arrayjob}
\usepackage{calc}






